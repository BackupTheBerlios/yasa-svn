\documentclass[11pt,a4paper]{report}
\usepackage{fullpage}

\author{G\"unther Bachfischer \\
        Harald Krau\ss{} \\
	Florian Kr\"usch \\
	Marc van Woerkom \\
        \\
        http://yasa.berlios.de}
       
\title{\textbf{YASA}\\
       {\Large Yet Another Simulated Annealing}\thanks{Developped during the 
       {\em Fachpraktikum 01597: Parallel Programming} 
       at FernUniversit\"at in Hagen, summer 2006}}

%\date{18.\ Mai 2006}


\begin{document}
\maketitle

\chapter{Introduction}

The YASA Project uses the {\em Simulated Annealing} metaheuristic to determine
near-optimal schedules for a given task graph with communication costs.
This metaheuristic was inferred from the domain of statistical physics,
where the process of annealing in solids can be interpreted as an optimization
procedure yielding thermal equilibrium \cite{Kirkpatrick1983}.

\section{Task Graphs}
A {\em task graph} is a directed graph $G = (V, E)$ where the set of
nodes $V$ consists of {\em tasks} and where there is an edge $e = (v_i, v_j)$
between two nodes $v_i$ and $v_j$ if task $v_i$ needs to have been computed 
before the computation of task $v_j$ may start, 
or short: if $v_j$ depends on $v_i$. 
Nodes are labeled with the time needed for computation.
Task graphs with {\em communication costs} have edges that are labeled with 
the time needed for transferring the computation results between the 
incident nodes, resulting in minimum time gaps between $v_i$ and $v_j$.

% image of task graph

\begin{verbatim}
# first.stg
        6       2
        0       3       0
        1       5       1
                                0       2
        2       2       1
                                0       1
        3       2       1
                                0       3
        4       4       1
                                0       4
        5       1       4
                                1       2
                                2       3
                                3       1
                                4       2
\end{verbatim}

\section{Schedules}

% Explanation

% image

% file


\section{Simulated Annealing}

According to \cite{Kirkpatrick1983} four ingredients are needed to simulate
annealing of a combinatorial optimization problem:
\begin{enumerate}
\item A concise description of a configuration of the system,
\item a random generator of {\em moves} or rearrangements of the elements
      in a configuration,
\item a quantitative objective function containing the trade--offs that
      have to be made and 
\item an annealing schedule of the temperatures and length of times for
      which the system is to be evolved. 
\end{enumerate} 

\begin{thebibliography}{99}

\bibitem{Kirkpatrick1983} Kirkpatrick, S., C.\ D.\ Gelatt, Jr.\ and M.\ P.\ Vecchi:
\emph{Optimization by Simulated Annealing},
Science, Volume~220, Number~4598, p.~671, 1983

\end{thebibliography}

\end{document}
