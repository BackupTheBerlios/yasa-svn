\documentclass[12pt,a4paper]{article}
\usepackage[german]{babel}
\usepackage{fullpage}

\author{G\"unther Bachfischer \\
        Harald Krau\ss{} \\
	Florian Kr\"usch \\
	Marc van Woerkom}
       
\title{\textbf{Monatsbericht Nr.\ 2}\\
       {\normalsize Gruppe Simulated Annealing} \\
       {\small Fachpraktikum 01597 Parallel Programming} \\
       {\small FernUniversit\"at in Hagen, SS 2006}}

\date{19.\ Juni 2006}


\begin{document}
\maketitle

\section{Organisatorisches}
Wir sind bis heute nicht sicher eine geeignete indirekte Repr"asentation
f"ur Schedules gefunden zu haben, welche eine effiziente N"aherungsl"osung
des Problems erm"oglicht, also insbesonders, dass Taskgraphen mit bis zu 
250 Tasks bew"altigt werden k"onnen.
Daher hatten wir uns vor ca.\ 2 Wochen in zwei Teams aufgeteilt: 
\begin{itemize}
\item Ein Team pirscht vor und leistet das an Vorarbeit
      f"ur die Parallelisierung, was ohne Detailkenntnis
      der Struktur von Konfigurationen und Suchraum m"oglich
      ist.
\item Ein Team versucht weiterhin eine geeignete Repr"asentation zu finden,
      um die sequentielle Implementierung der SA Metaheuristik 
      abzuschliessen. 
\end{itemize}

\section{Struktur von Konfigurationen und Suchraum}
Es ist unstrittig, dass nicht alle Informationen eines Schedules
(direkte Repr"asentation) f"ur die Suche mit Hilfe der SA Metaheuristik
ben"otigt werden, vielmehr ist eine Teilmenge dieser Daten gefragt,
evt. in transformierter Form, auf der effektiv mutiert und
gesucht werden kann und aus der erst gegen Ende des Verfahrens
die direkte Repr"asentation gewonnen werden kann.\\

\noindent
Hier sind wir zun"achst etwas zu radikal gewesen: Die Annahme war,
dass das Gesamtproblem in zwei Teile zerf"allt:
\begin{enumerate}
\item Der Bestimmung der Abarbeitungsreihenfolge (also eine 
      Liste von Tasks) und
\item der Zuordnung der Tasks auf die Prozessoren (sog.\ Mapping),
\end{enumerate}
wobei wir davon ausgegangen sind, dass das Mapping automatisch (durch
eine Heuristik, wie z.B. Zuteilung auf den n"achsten freien Prozessor, oder
zuf"alliges Zuteilen auf einen freien Prozessor) erfolgen k"onnte.
Die indirekte Repr"asentation h"atte damit lediglich aus einer
Liste von Tasknummern bestanden. 
Dieser Ansatz wird z.B.\ in \cite{DavidovicHansenMladenovic2001}
verfolgt. \\

\noindent
Uns ist dann klar geworden, dass die Metaheuristik damit
im wesentlichen auf die Reihenfolge, nicht aber auf das
Mapping wirken w"urde. Ebenso hatten wir Hinweise von der
Betreuung erhalten, uns st"arker mit der Literatur zu besch"aftigen,
insb.\ \cite{KwokAhmad1999}. \\

\noindent
Wir halten aus diesem Artikel folgende Stelle f"ur uns relevant:
\begin{quote}
Hou et al.\ \cite{Hou1994} also proposed a
scheduling algorithm using a genetic
search in which each chromosome is a
collection of lists, and each list represents
the schedule on a distinct processor.
Thus, each chromosome is not a
linear structure but a two-dimensional
structure instead. One dimension is a
particular processor index and the other
is the ordering of tasks scheduled on the
processor. \\ 
Using such an encoding
scheme poses a restriction on the schedules
being represented: the list of tasks
within each processor in a schedule is
ordered in ascending order of their topological
height, which is defined as the
largest number of edges from an entry
node to the node itself. \\
(..) \\
In a mutation, two
randomly chosen tasks with the same
height are swapped in the schedule. As
to the generation of the initial population,
$N_p$ randomly permuted schedules
obeying the height ordering restriction
are generated.
\end{quote}
Eine Konfiguration w"urde hier aus $N_p$ vielen Tasklisten
bestehen, wobei $N_p$ die Anzahl der Prozessoren ist.
Damit w"are auch das Mapping der SA Metaheuristik unterworfen. \\

\noindent 
Offensichtliche elementare Mutationen w"aren horizontale Verschiebungen
(innerhalb der Taskliste eines Prozessors) oder vertikale Verschiebungen
(in die Taskliste eines anderen Prozessors). 
Z.B. k"onnte man eine Mutation als Paar $(p', t')$ beschreiben, wobei $p'$
die Nummer des neuen Prozessors und $t'$ die Platznummer innerhalb
der dortigen Taskliste ist, wo der Task eingef"ugt werden soll. 
Die einfachste Strategie w"are diese beiden Zahlen einfach zu w"urfeln.
Dabei ist jeweils zu pr"ufen, ob die neue Position zul"assig ist, also 
mit dem Taskgraphen vertr"aglich. Ein alternativer Ansatz
w"are, wie im obigen Zitat aus \cite{Hou1994}, die neuen Positionen von
vornherein auf g"ultige zu beschr"anken, dort mit Hilfe der topologischen
H"ohe eines Tasks. \\

\noindent
Als Abstand schlagen wir die Anzahl der Mutationen vor.
Diesen Ansatz findet man auch in \cite{DavidovicHansenMladenovic2001} erw"ahnt, dort
spricht man von $k$-opt Nachbarschaftsstrukturen. \\

\noindent
Aktuell implementieren wir die SA Metaheuristik mit o.g.\ indirekter Repr"asentation.

\section{Parallelisierung}
F"ur die Parallelisierung des SA-Verfahrens wird gerade der Rahmen 
geschaffen. \\

\noindent
Da f"ur alle drei L"osungsans"atze entweder der gesamte L"osungsraum 
oder eine Teilmenge davon zu "ubertragen ist, stellt dies auch den ersten 
Schritt bei der Umsetzung dar, d.h.\ die sichere Daten"ubertragung zwischen
Master und den Slaves bzw.\ umgekehrt. 
Der Ansatz ist dabei so allgemein zu halten, dass er f"ur alle drei Verfahren 
dienen kann. \\

\noindent
F"ur die Variante mit den disjunkten Teilbereichen wird derzeit ein 
Master/Slave-Ansatz implementiert.

\begin{thebibliography}{99}
\bibitem{DavidovicHansenMladenovic2001}
Davidovic, Tatjana und Hansen, Pierre und Mladenovic, Nenad:
\emph{Variable Neighborhood Search for Multiprocessor Scheduling Problem with Communication Delays},
Proceedings MIC'2001 - $4^{\mbox{th}}$ Metaheuristics International Conference, 
S.~737--741, Porto, Portugal,
2001

\bibitem{Hou1994}
Hou, E.S.H.\ und Ansari, N.\ und Ren, H.: 
\emph{A genetic algorithm for multiprocessor scheduling},
IEEE Trans.\ Parallel Distrib.\ Syst.\ 5, 2 (Feb.), S.~113--120, 1994

\bibitem{KwokAhmad1999} 
Kwok, Yu-Kwong und Ahmad, Ishfaq: 
\emph{Static Scheduling Algorithms for Allocating Directed Task Graphs to Multiprocessors},
ACM Computing Surveys, Vol.~31, No.~4, December~1999

\end{thebibliography}


\end{document}
