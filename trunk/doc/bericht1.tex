\documentclass[12pt,a4paper]{article}
\usepackage[german]{babel}
\usepackage{fullpage}

\author{G\"unther Bachfischer \\
        Harald Krau\ss{} \\
	Florian Kr\"usch \\
	Marc van Woerkom}
       
\title{\textbf{Monatsbericht Nr.\ 1}\\
       {\normalsize Gruppe Simulated Annealing} \\
       {\small Fachpraktikum 01597 Parallel Programming} \\
       {\small FernUniversit\"at in Hagen, SS 2006}}

\date{18.\ Mai 2006}


\begin{document}
\maketitle

\section{Gruppenbildung}
Die Gruppenbildung erfolgte am 1.\ Pr"asenztermin in Hagen, am 7.\ April 2006.
Ausschlaggebend f"ur die Zusammensetzung war einerseits das Interesse der vier 
Mitglieder am speziellen Thema {\em Simulated Annealing}, andererseits sprach
die relative N"ahe der Wohnorte daf"ur, denn drei der Mitglieder wohnen in 
Bayern, sowie zwei im Gro"sraum 
K"oln/D"usseldorf\footnote{Ein Mitglied wohnt verteilt.},
was eventuelle pers"onliche Treffen erleichtert.
  
\section{Arbeitsorganisation}
Am Pr"asenztermin wurden die Kontaktdaten auf einem Zettel erfasst, und 
daher konnte direkt im Anschluss die gemeinsame Arbeit zun"achst per
e-Mail organisiert werden.
\noindent
Rasch wurden grundlegende Entscheidungen per Abstimmung getroffen:
\begin{itemize}
\item Der Name unseres Projektes lautet {\em YASA} (Yet Another Simulated
      Annealing).
\item Unser Projekt wird als Open Source ver"offentlicht, dazu verwenden 
      wir die BSD-Lizenz (in der abgeschw"achten modernen Form, 
      d.h.\ insbesondere ohne Werbeklausel).
\item Wir nutzen die Infrastruktur von Berlios f"ur unser Projekt.
      Dies beinhaltet Webserver, Shell-Account, Mailinglisten, 
      Versionsverwaltung, Bugtracker und weitere Dienste.
\item Wir verwenden das modernere Subversion anstatt CVS zur Verwaltung 
      der Projektdateien.
\item Die weitere Kommunikation l"auft "uber die (Projekt-interne) 
      Mailingliste yasa-dev@berlios.de. 
      F"ur die commit-logs wurde die Mailingliste 
      yasa-svn@berlios.de eingerichtet.
\item F"ur gemeinsame Diskussionen wurden IRC Chat Sitzungen 
      auf irc://irc.fernuni-hagen.de abgehalten.
      Die Chats wurden mitgeschnitten und der Mitschnitt jeweils
      in der yasa-dev Mailingliste ver"offentlicht.
\end{itemize}

\section{Theoretische "Uberlegungen}
Wir haben uns die empfohlene Literatur durchgelesen und uns 
erste L"osungsstrategien erarbeitet.
Offene Fragen wurden untereinander oder mit dem Betreuer,
Herrn H"onig, gekl"art.

\section{Praktische Arbeiten}
Ein Quellcodebaum wurde im Subversion repository auf Berlios
angelegt.
Das C Programm ist in der Lage stg-Dateien zu lesen und
ssf-Dateien zu schreiben.
Ein Prototyp f"ur ein L"osungsverfahren wurde in JavaScript
implementiert.

\end{document}
